\htmlhr
\chapter{Advanced type system features\label{advanced-type-system-features}}

This chapter describes features that are automatically supported by every
checker written with the Checker Framework.
You may wish to skim or skip this chapter on first reading.  After you have
used a checker for a little while and want to be able to express more
sophisticated and useful types, or to understand more about how the Checker
Framework works, you can return to it.


\section{Invariant array types\label{invariant-arrays}}

Java's type system is unsound with respect to arrays.  That is, the Java
type-checker approves code that is unsafe and will cause a run-time crash.
Technically, the problem is that Java has ``covariant array types'', such
as treating \<String[]> as a subtype of \<Object[]>.  Consider the
following example:

\begin{Verbatim}
  String[] strings = new String[] {"hello"};
  Object[] objects = strings;
  objects[0] = new Object();
  String myString = strs[0];
\end{Verbatim}

\noindent
The above code puts an \<Object> in the array \<strings> and thence in
\<myString>, even though \<myString = new Object()> should be, and is,
rejected by the Java type system.  Java prevents corruption of the JVM by
doing a costly run-time check at every array assignment; nonetheless, it is
undesirable to learn about a type error only via a run-time crash rather
than at compile time.

When you pass the \<-AinvariantArrays> command-line option,
the Checker Framework is stricter than Java, in the sense that it treats
arrays invariantly rather than covariantly.  This means that a type system
built upon the Checker Framework is sound:  you get a compile-time
guarantee without the need for any run-time checks.  But it also means that
the Checker Framework rejects code that is similar to what Java unsoundly
accepts.  The guarantee and the compile-time checks are about your
extended type system.  The Checker Framework does not reject the example
code above, which contains no type annotations.

Java's covariant array typing is sound if the array is used in a read-only
fashion:  that is, if the array's elements are accessed but the array is
not modified.  However, fact about read-only usage is not built into any of
the type-checkers except those that are specifically about immutability:
IGJ (see \chapterpageref{igj-checker}) and Javari (see
\chapterpageref{javari-checker}).  Therefore, when using other type systems
along with \<-AinvariantArrays>, you will need to suppress any warnings that
are false positives because the array is treated in a read-only way.

% An alternative would be for the Nullness Checker to read and trust (but not
% verify) existing IGJ/Javari annotations; this would be unsound but would be
% a better and more explicit way of suppressing the warning.



\section{Context-sensitive type inference for array constructors\label{array-context-sensitive}}

When you write an expression, the Checker Framework gives it the most
precise possible type, depending on the particular expression or value.
For example, when using the Regex Checker (\chapterpageref{regex-checker}),
the string \<"hello"> is given type \<@Regex String> because it is a legal
regular expression (whether it is meant to be used as one or not) and the
string \<"(foo"> is given the type \<@Unqualified String> because it is not
a legal regular expression.

Array constructors work differently.  When you create an array with the
array constructor syntax, such as the right-hand side of this assignment:

\begin{Verbatim}
String[] myStrings = {"hello"};
\end{Verbatim}

\noindent
then the expression does not get the most precise possible type, because
doing so could cause inconvenience.  Rather, its type is determined by the
context in which it is used:  the left-hand side if it is in an assignment,
the declared formal parameter type if it is in a method call, etc.

In particular, if the expression \verb|{"hello"}| were given the type
\<@Regex String[]>, then the assignment would be illegal!  But the Checker
Framework gives the type \<String[]> based on the assignment context, so the code
type-checks.

If you prefer a specific type for a constructed array, you can indicate
that either in the context (change the declaration of \<myStrings>) or in a
\<new> construct (change the expression to \<new @Regex String[] >\verb|{"hello"}|).


\section{The effective qualifier on a type (defaults and inference)\label{effective-qualifier}}

A checker sometimes treats a type as having a slightly different qualifier
than what is written on the type --- especially if the programmer wrote no
qualifier at all.
Most readers can skip this section on first reading, because you will
probably find the system simply ``does what you mean'', without forcing
you to write too many qualifiers in your program.
In particular, qualifiers in method bodies are extremely rare.

Most of this section is applicable only to source code that is being
checked by a checker.
%
When the compiler reads a \<.class> file that was checked by a
checker, the \<.class> file contains the explicit or defaulted
annotations from the source code and no defaulting is necessary.
%
When the compiler reads a \<.class> file that was not checked by a
checker, the \<.class> file contains only explicit annotations and
defaulting might be necessary; see Section~\ref{defaults-classfile}
for these rules.

  The following steps determine the effective
qualifier on a type --- the qualifier that the checkers treat as being present.

\begin{enumerate}
\item
  If a type qualifier is present in the source code, that qualifier is used.

\item
  The type system adds implicit qualifiers.  This happens whether or not
  the programmer has written an explicit type qualifier.

  Here are some examples of implicit qualifiers:

\begin{itemize}
\item
  In the Nullness type system (see \chapterpageref{nullness-checker}),
  \<enum> values, string literals, and method receivers are always non-null.
\item
  In the Interning type system (see \chapterpageref{interning-checker}),
  string literals and \<enum> values are always interned.
\end{itemize}

  If the type has an implicit qualifier, then it is an error to write an
  explicit qualifier that is equal to (redundant with) or a supertype of
  (weaker than) the implicit qualifier.  A programmer may strengthen
  (write a subtype of) an implicit qualifier, however.

  Implicit qualifiers arise from two sources:
\begin{description}
\item[built-in]
  Implicit qualifiers can be
  built into a type system (Section~\ref{writing-type-introduction}), in
  which case the type system's documentation explains all of the type
  system's implicit qualifiers.  Both of the above examples are built into
  the Nullness type system.
\item[programmer-declared]
  A programmer may introduce an implicit annotation on each use of class
  $C$ by writing a qualifier on the declaration of class $C$.  If \<MyClass>
  is declared as \<class @MyAnno MyClass \ttlcb...\ttrcb>, then each occurrence of
  \<MyClass> in the source code is treated as if it were \<@MyAnno
  MyClass>.
\end{description}



\item
  If there is no explicit or implicit qualifier on a type, then a default
  qualifier
  % except for type parameters, but don't clutter this text with that detail
  is applied; see Section~\ref{defaults}.

%BEGIN LATEX
  % This looks bad in HTML.
  \smallskip
%END LATEX

  At this point (after step 3), every type has a qualifier.

\item
  The type system may refine a qualified type on a local variable --- that
  is, treat it as a subtype of how it was declared or defaulted.  This
  refinement is always sound and has the effect of eliminating false
  positive error messages.  See Section~\ref{type-refinement}.

  % Type
  % qualifier refinement is implemented by the \refclass{framework/flow}{Flow} class.

\end{enumerate}



\subsection{Default qualifier for unannotated types\label{defaults}}

A type system designer, or an end-user programmer, can cause unannotated
references to be treated as if they had a default annotation.

There are several defaulting mechanisms, for convenience and flexibility.
When determining the default qualifier for a use of a type, the following
rules are used in order, until one applies.
\begin{itemize}
\item
  Use the innermost user-written \code{@DefaultQualifier}, as explained in
  this section.
\item
  Use the default specified by the type system designer
  (Section~\ref{typesystem-defaults});
  this is usually CLIMB-to-top (Section~\ref{climb-to-top}).
\item
  Use \refqualclass{framework/qual}{Unqualified}, which the framework
  inserts to avoid ambiguity and simplify the programming interface for
  type system designers.  Users do not have to worry about this detail,
  but type system implementers can rely on the fact that some
  qualifier is present.
\end{itemize}

% (Implementation detail:  setting defaults is implemented by the
% \refclass{framework/util}{QualifierDefaults} class.)


The end-user programmer specifies a default qualifier by writing the \refqualclass{framework/qual}{DefaultQualifier}
annotation on a package, class, method, or variable declaration.  The
argument to \refqualclass{framework/qual}{DefaultQualifier} is the \code{String}
name of an annotation.  It may be a short name like \code{"NonNull"}, if an
appropriate import statement exists.  Otherwise, it should be
fully-qualified, like \code{"org.checkerframework.checker.nullness.qual.NonNull"}.
The optional second argument indicates where the default
applies.  If the second argument is omitted, the specified annotation is
the default in all locations.  See the Javadoc of \refclass{framework/qual}{DefaultQualifier} for details.

For example, using the Nullness type system (Chapter~\ref{nullness-checker}):

\begin{Verbatim}
import org.checkerframework.framework.qual.*;        // for DefaultQualifier[s]
import org.checkerframework.checker.nullness.qual.NonNull;

@DefaultQualifier(NonNull.class)
class MyClass {

  public boolean compile(File myFile) { // myFile has type "@NonNull File"
    if (!myFile.exists())          // no warning: myFile is non-null
      return false;
    @Nullable File srcPath = ...;  // must annotate to specify "@Nullable File"
    ...
    if (srcPath.exists())          // warning: srcPath might be null
      ...
  }

  @DefaultQualifier(Mutable.class)
  public boolean isJavaFile(File myfile) {  // myFile has type "@Mutable File"
    ...
  }
}
\end{Verbatim}

If you wish to write multiple
\refqualclass{framework/qual}{DefaultQualifier} annotations at a single location,
use
\refqualclass{framework/qual}{DefaultQualifiers} instead.  For example:

\begin{Verbatim}
@DefaultQualifiers({
  @DefaultQualifier(NonNull.class),
  @DefaultQualifier(Mutable.class)
})
\end{Verbatim}


If \code{@DefaultQualifier}[\code{s}] is placed on a package (via the
\<package-info.java> file), then it applies to the given package \emph{and}
all subpackages.
% This is slightly at odds with Java's treatment of packages of different
% names as essentially unrelated, but is more intuitive and useful.

Recall that an annotation on a class definition indicates an implicit
qualifier (Section~\ref{effective-qualifier}) that can only be
strengthened, not weakened.  This can lead to unexpected results if
the default qualifier applies to a class definition.  Thus, you may want to
put explicit qualifiers on class declarations (which prevents the default
from taking effect), or exclude class declarations from defaulting.


%% Don't even bother to bring this up; it will just sow confusion without
%% being helpful.
% For some type systems, a user may not specify a default qualifier, or doing
% so prevents giving any other qualifier to any reference.  This is a
% consequence of the design of the type system; see
% Section~\ref{bottom-and-top-qualifier}.


When a programmer omits an \<extends> clause at a declaration of a type
parameter, the default still applies to the implicit upper bound.  For
example, consider these two declarations:

\begin{Verbatim}
  class C<T> { ... }
  class C<T extends Object> { ... }  // identical to previous line
\end{Verbatim}

\noindent
The two declarations are treated identically by Java, and the default
qualifier applies to the \<Object> upper bound whether it is implicit or
explicit.  (The @NonNull default annotation applies only to the upper bound
in the \<extends> clause, not to the lower bound in the inexpressible
implicit \<super void> clause.)


\subsection{Defaulting rules and CLIMB-to-top\label{climb-to-top}}

Each type system defines a default qualifier.  For example, the default
qualifier for the Nullness Checker is
\refqualclass{checker/nullness/qual}{NonNull}.  That means that when a user
writes a type such as \<Date>, the Nullness Checker interprets it as
\<@NonNull Date>.

We recommend that the type system apply that default qualifier to most but
not all types.  In particular, we recommend the CLIMB-to-top rule.  This
rule states that the \emph{top} qualifier in the hierarchy is applied to
the CLIMB locations:  \textbf{C}asts, \textbf{L}ocals, \textbf{I}nstanceof, and i\textbf{M}plicit \textbf{B}ounds.
For example, when the user writes a type such as \<Date> in such a
location, the Nullness Checker interprets it as \<@Nullable Date> (because
\refqualclass{checker/nullness/qual}{Nullable} is the top qualifier in the
hierarchy, see Figure~\ref{fig-nullness-hierarchy}).

% TODO:  Add wildcard bounds to the set of type-refined locations?
% Especially in light of
% https://github.com/typetools/checker-framework/issues/260

The CLIMB-to-top rule is used only for unannotated source code that is
being processed by a checker.  For unannotated libraries (code read by the
compiler in \<.class> or \<.jar> form), the checker uses conservative
defaults (Section~\ref{defaults-classfile}).

The rest of this section explains the rationale and implementation of
CLIMB-to-top.

Here is the rationale for CLIMB-to-top:

\begin{itemize}
\item
Casts and local variables (including resource variables in the
try-with-resources construct, variables in for statements, exception parameters etc.)
should be defaulted to top because they are the locations to
which type refinement (Section~\ref{type-refinement}) is applied.  If they
start as the top type, then the Checker Framework chooses the best (most
general) possible type for them.  As a result, a programmer rarely writes
an explicit annotation on any of those locations.

Catch arguments, known as exception parameters in the Java Language Specification,
should be defaulted to top for most checkers; otherwise, an error will be issued.  This is because
exceptions of arbitrary qualified types can be thrown and the Checker Framework
does not provide runtime checks.


\item
Instanceof types are defaulted to top for a similar reason:  so that
programmers do not need to write annotations on them.  If the
\<instanceof>'s qualifier is top, then the Checker Framework will never
issue an error that the \<instanceof> qualifier is not compatible with the
argument.

\item
Implicit upper bounds are defaulted to top to allow them to be instantiated
in any way.  If a user declared \code{class C<T> \ttlcb\ ... \ttrcb}, then
we assume that the user intended to allow any instantiation of the class,
and the declaration is interpreted as \code{class C<T extends @Nullable
  Object> \ttlcb\ ... \ttrcb} rather than as \code{class C<T extends
  @NonNull Object> \ttlcb\ ... \ttrcb}.  The latter would forbid
instantiations such as \code{C<@Nullable String>}, or would require
rewriting of code.  On the other hand, if a user writes an explicit bound
such as \code{class C<T extends D> \ttlcb\ ... \ttrcb}, then the user
intends some restriction on instantiation and can write a qualifier on the
upper bound as desired.

This rule means that the upper bound of \code{class C<T>} is defaulted
differently than the upper bound of \code{class C<T extends Object>}.  It
would be more confusing for ``\code{Object}'' to be defaulted differently in \code{class C<T extends Object>} and in an
instantiation \code{C<Object>}, and for the upper bounds to be defaulted
differently in \code{class C<T extends Object>}
and \code{class C<T extends Date>}.

\item
Implicit \emph{lower} bounds are defaulted to the bottom type, again to allow
maximal instantiation.  Note that Java does not allow a programmer to
express both the upper and lower bounds of a type, but the Checker
Framework allows the programmer to specify either or both;
see Section~\ref{generics-defaults}.

\end{itemize}

Here is how the CLIMB-to-top rule is expressed for the Nullness Checker:

\begin{Verbatim}
@DefaultQualifierInHierarchy
public @interface NonNull { }

@DefaultFor({ DefaultLocation.LOCAL_VARIABLE, DefaultLocation.RESOURCE_VARIABLE,
    DefaultLocation.IMPLICIT_UPPER_BOUNDS })
public @interface Nullable { }
\end{Verbatim}

\noindent
Note that \<DefaultLocation.LOCAL\_VARIABLE> includes casts and instanceof.
 As mentioned above, the exception parameters are always non-null, so
\<DefaultLocation.EXCEPTION\_PARAMETER> is excluded from the above list.

A type system designer does not have to use the CLIMB-to-top rule.  In
addition, a user may choose a different rule for defaults using the
\refqualclass{framework/qual}{DefaultQualifier} annotation; see
Section~\ref{defaults}.


\subsection{Inherited defaults\label{inherited-defaults}}

In certain situations, it would be convenient for an annotation on a
superclass member to be automatically inherited by subclasses that override
it.  This feature would reduce both annotation effort and program
comprehensibility.  In general, a program is read more often than it is
edited/annotated, so the Checker Framework does not currently support this
feature.  Here are more detailed justifications:

\begin{itemize}

\item
  Currently, a user can determine the annotation on a parameter or return
  value by looking at a single file.  If annotations could be inherited
  from supertypes, then a user would have to examine all supertypes to
  understand the meaning of an unannotated type in a given file.

\item
  Different annotations might be inherited from a supertype and an
  interface, or from two interfaces.  Presumably, the subtype's annotations
  would be stronger than either (the greatest lower bound in the type
  system), or an error would be thrown if no such annotations existed.

\end{itemize}

If these issues can be resolved, then the feature may be added in the
future.  Or, it may be added optionally, and each type-checker
implementation can enable it if desired.


\subsection{Inherited wildcard annotations\label{inherited-wildcard-annotations}}

If a wildcard is unbounded and has no annotation (e.g. \code{List<?>}),
the annotations on the wildcard's bounds are copied from the type parameter
to which the wildcard is an argument.  For example, the two wildcards in
the declarations below are equivalent.

\begin{Verbatim}
class MyList<@Nullable T extends @Nullable Object> {}

MyList<?> listOfNullables;
MyList<@Nullable ? extends @Nullable Object> listOfNullables;
\end{Verbatim}

We copy these annotations because wildcards must be within the bounds of their
corresponding type parameter.  Therefore, there would be many false positive
\code{type.argument.type.incompatible} warnings if the bounds of a wildcard
were defaulted differently from the bounds of its corresponding type parameter.
Here is another example:

\begin{Verbatim}
class MyList<@Regex(5) T extends @Regex(1) Object> {}

MyList<?> listOfRegexes;
MyList<@Regex(5) ? extends @Regex(1) Object> listOfRegexes;
\end{Verbatim}

Note, this applies only to unbounded wildcards.  The two wildcards in the
following example are equivalent.

\begin{Verbatim}
class MyList<@Nullable T extends @Nullable Object> {}

List<? extends Object> listOfNonNulls;
List<@NonNull ? extends @NonNull Object> listOfNonNulls2;
\end{Verbatim}

Note, the upper bound of the wildcard \code{? extends Object} is defaulted to
\code{@NonNull} using the CLIMB-to-top rule (see Section~\ref{climb-to-top}).

\subsection{Default qualifiers for \<.class> files (conservative library defaults)\label{defaults-classfile}}

\emph{Note:} For release 1.9.3, the conservative library defaults presented in this section
are off by default and can be turned on by supplying the \<-AsafeDefaultsForUnannotatedBytecode>
command-line option.  Starting with release 1.9.4, they will be turned on
by default and it will be possible to turn them off by supplying a
\<-AunsafeDefaultsForUnannotatedBytecode> command-line option.
That option will replace the existing \<-AsafeDefaultsForUnannotatedBytecode>
command-line option.

The defaulting rules presented so far apply to source code that is read by
the compiler.  When the compiler reads a \<.class> file, different
defaulting rules apply.

If the checker was run during the compiler execution that created the
\<.class> file,
%% This caveat is true, but it's a distraction at this point in the manual.
%% Also, not having a top and a bottom qualifier is an uncommon case.
% (and the qualifier hierarchy has both a top and a bottom
% qualifier, see Section~\ref{bottom-and-top-qualifier}),
then there is no need for
defaults:  the \<.class> file has an explicit qualifier at each type use.
(Furthermore, unless warnings were suppressed, those qualifiers are
guaranteed to be correct.)
When you are performing pluggable type-checking,
it is best to ensure that the compiler only reads such \<.class> files.
Section~\ref{compiling-libraries} discusses how to create annotated
libraries.
%% True, but not relevant to the point of this paragraph.
% , even if the library source code is only partially annotated.

If the checker was not run during the compiler execution that created the
\<.class> file, then the \<.class> file contains only the type qualifiers
that the programmer wrote explicitly.  (Furthermore, there is no guarantee
that these qualifiers are correct, since they have not been checked.)
In this case, each checker decides what qualifier to use for the
locations where the programmer did not write an annotation.  The typical
choice is:

\begin{itemize}
\item
  For method parameters, use the bottom qualifier (see
  Section~\ref{bottom-qualifier}).
\item
  For method return values, use the top qualifier (see
  Section~\ref{top-qualifier}).
\end{itemize}

For example, an unannotated method

\begin{Verbatim}
  String concatenate(String p1, String p2)
\end{Verbatim}

\noindent
in a classfile would be interpreted as

\begin{Verbatim}
  @Top String concatenate(@Bottom String p1, @Bottom String p2)
\end{Verbatim}

There is no single possible default that is sound for fields.  In the rare
circumstance that there is a mutable public field in an unannotated
library, the Checker Framework may fail to warn about code that can
misbehave at run time.  The Checker Framework developers are working to
improve handling of mutable public fields in unannotated libraries.

%% TODO: The following rule is preferable to the current safe behavior.
%% However, we have not yet figured out how to implement separate handling
%% for field read vs. write.
% For fields, use the bottom qualifier when writing to the field and the
% top qualifier when reading from the field.

%% TODO: should give rules for other locations, such as type parameters
%% (which should behave like fields), bounds, etc.

These choices are conservative.  They are likely to cause many
false-positive type-checking errors, which will help you to know which
library methods need annotations.  You can then write those library
annotations (see Chapter~\ref{annotating-libraries}) or alternately
suppress the warnings (see Section~\ref{suppressing-warnings}).

% If you supply the command-line option
% \<-AunsafeDefaultsForUnannotatedBytecode>,
% then the checker does defaulting for unannotated bytecode like it does for
% annotated source code.  In other words, when a type use in a \<.class> file
% has no explicit annotation, it is defaulted using the same rules as for the
% corresponding source code location.  You should only use this command-line
% option as temporary measure, because it is unsafe:  the checker might issue
% no warnings even though the code could violate the type guarantee at run
% time.  However, it can be useful when you are first annotating a codebase,
% to help you focus on errors within the codebase before you have annotated
% external libraries.

% TODO: document
% the \refqualclass{framework/qual}{DefaultForUnannotatedCode}
% and \refqualclass{framework/qual}{DefaultQualifierForUnannotatedCode}
% annotations for type system designers.


\section{Automatic type refinement (flow-sensitive type qualifier inference)\label{type-refinement}}

The
checkers soundly treat certain variables and expressions as having a
subtype of their declared or defaulted (Section~\ref{defaults})
type.
This functionality
reduces your burden of annotating types in your program and
eliminates some false positive warnings, but it
never introduces unsoundness nor causes an error to be missed.

By default all checkers, 
automatically incorporate type refinement.  Most of the time, users don't
have to think about, and may not even notice, type refinement.
(And most readers can skip reading this section of the manual, except
possibly the examples in Section~\ref{type-refinement-examples}.)
The
checkers simply do the right thing even when a programmer omits an
annotation on a local variable, or when a programmer writes an
unnecessarily general type in a declaration.

The functionality has a variety of names:  automatic type refinement,
flow-sensitive type qualifier inference, local type inference, and
sometimes just ``flow''.

If you find examples where you think a value should be inferred to have
(or not have) a
given annotation, but the checker does not do so, please submit a bug
report (see Section~\ref{reporting-bugs}) that includes a small piece of
Java code that reproduces the problem.


\subsection{Type refinement examples\label{type-refinement-examples}}

Suppose you write

\begin{Verbatim}
  @Nullable String myVar;
  ...
  if (myVar != null) {
    myVar.hashCode();
  }
\end{Verbatim}

\noindent
The Nullness Checker issues a warning whenever a method such as
\<hashCode()> is called on a possibly-null value, which  may result in a
null pointer exception.
However, the Nullness Checker does not issue a warning for the call
\<myVar.hashCode()>
in the code above.
Within the body
of the \<if> test, the type of \<myVar> is \<@NonNull String>, even though
\<myVar> is declared as \<@Nullable String>.

Here is another example:

\begin{Verbatim}
  @Nullable String myVar;
  ...                   // myVar has type @Nullable String
  myVar = "hello";
  ...                   // myVar has type @NonNull String
  myVar.hashCode();
  ...
  myVar = myMap.get(someKey);
  ...                   // myVar has type @Nullable String
\end{Verbatim}

\noindent
The Nullness Checker does not issue a warning for
the call <myVar.hashCode()> above because after the assignment,
the type-checker
treats \<myVar> as having type \<@NonNull String>, which is more precise
than the programmer-written type.


Flow-sensitive type refinement applies to every checker, including new
checkers that you write.  Here is an example for the Regex Checker
(\chapterpageref{regex-checker}):

\begin{Verbatim}
  void m2(@Unannotated String s) {
    s = RegexUtil.asRegex(s, 2);  // asRegex throws error if arg is not a regex
                                  // with the given number of capturing groups
    ...   // s now has type "@Regex(2) String"
  }
\end{Verbatim}


As a further example, 
consider this code, along with comments indicating whether the
Nullness Checker (Chapter~\ref{nullness-checker}) issues a warning.  Note that the same expression may yield a
warning or not depending on its context.

\begin{Verbatim}
  // Requires an argument of type @NonNull String
  void parse(@NonNull String toParse) { ... }

  // Argument does NOT have a @NonNull type
  void lex(@Nullable String toLex) {
    parse(toLex);        // warning:  toLex might be null
    if (toLex != null) {
      parse(toLex);      // no warning:  toLex is known to be non-null
    }
    parse(toLex);        // warning:  toLex might be null
    toLex = new String(...);
    parse(toLex);        // no warning:  toLex is known to be non-null
  }
\end{Verbatim}

This example shows the general rules for when
the Nullness Checker (Chapter~\ref{nullness-checker}) can automatically
determine that certain variables are non-null, even if they were explicitly
or by default annotated as nullable.
The checker treats a variable or expression as \refqualclass{checker/nullness/qual}{NonNull}
\begin{itemize}
\item
starting at the time that it is either
assigned a non-null value or checked against null (e.g., via an assertion,
\code{if} statement, or being dereferenced)
\item
until it might be re-assigned (e.g.,
via an assignment that might affect this variable, or via a method call
that might affect this variable).
\end{itemize}


The inference indicates when a variable can be treated as having a subtype
of its declared type --- for instance, when an otherwise nullable type can be
treated as a \refqualclass{checker/nullness/qual}{NonNull} one.  The inference never treats a variable as
a supertype of its declared type (e.g., an expression with declared type \refqualclass{checker/nullness/qual}{NonNull}
type is never inferred to be treated as possibly-null).



\subsection{Types that are not refined\label{type-refinement-types-that-are-not-refined}}

Array element types and generic arguments are never changed by type
refinement.  Changing these components of a type never yields a subtype of
the declared type.
For example, \code{List<Number>} is \emph{not} a
subtype of \code{List<Object>}.  Similarly, the Checker Framework does not
treat \code{Number[]} as a subtype of
\code{Object[]}.
For details, see Section~\ref{covariant-type-parameters} and
Section~\ref{invariant-arrays}.


\subsection{Run-time tests and type refinement\label{type-refinement-runtime-tests}}

Some type systems support a run-time test that the Checker Framework can
use to refine types within the scope of a conditional such as \<if>, after
an \<assert> statement, etc.

Whether a type system supports such a run-time test depends on whether the
type system is computing properties of data itself, or properties of
provenance (the source of the data).  An example of a property about data is
whether a string is a regular expression.  An example of a property about
provenance is units of measure:  there is no way to look at the
representation of a number and determine whether it is intended to
represent kilometers or miles.

% Keep these lists in sync with the list in introduction.tex

Type systems that support a run-time test are:
\begin{itemize}
\item
  \ahrefloc{nullness-checker}{Nullness Checker} for null pointer errors
  (see \chapterpageref{nullness-checker})
\item
  \ahrefloc{map-key-checker}{Map Key Checker} to track which values are
  keys in a map (see \chapterpageref{map-key-checker})
\item
  \ahrefloc{lock-checker}{Lock Checker} for concurrency and lock errors
  (see \chapterpageref{lock-checker})
\item
  \ahrefloc{regex-checker}{Regex Checker} to prevent use of syntactically
  invalid regular expressions (see \chapterpageref{regex-checker})
\item
  \ahrefloc{formatter-checker}{Format String Checker} to ensure that format
  strings have the right number and type of \<\%> directives (see
  \chapterpageref{formatter-checker})
\item
  \ahrefloc{i18n-formatter-checker}{Internationalization Format String Checker}
  to ensure that i18n format strings have the right number and type of
  \<\{\}> directives (see \chapterpageref{i18n-formatter-checker})
\end{itemize}


Type systems that do not currently support a run-time test, but could do so with some
additional implementation work, are

\begin{itemize}
\item
  \ahrefloc{interning-checker}{Interning Checker} for errors in equality
  testing and interning (see \chapterpageref{interning-checker})
\item
  \ahrefloc{propkey-checker}{Property File Checker} to ensure that valid
  keys are used for property files and resource bundles (see
  \chapterpageref{propkey-checker})
\item
  \ahrefloc{i18n-checker}{Internationalization Checker} to
  ensure that code is properly internationalized (see
  \chapterpageref{i18n-checker})
% The Compiler Message Key Checker is neither here nor in the introduction
% chapter because it is really meant for Checker Framework developers and
% as sample code, and is not meant for Checker Framework users at large.
\item
  \ahrefloc{signature-checker}{Signature String Checker} to ensure that the
  string representation of a type is properly used, for example in
  \<Class.forName> (see \chapterpageref{signature-checker}).
\item
  \ahrefloc{constant-value-checker}{Constant Value Checker} to determine
  whether an expression's value can be known at compile time
  (see \chapterpageref{constant-value-checker})
\end{itemize}


Type systems that cannot support a run-time test are:

\begin{itemize}
\item
  \ahrefloc{initialization-checker}{Initialization Checker} to ensure all
  fields are set in the constructor (see
  \chapterpageref{initialization-checker})
\item
  \ahrefloc{fenum-checker}{Fake Enum Checker} to allow type-safe fake enum
  patterns (see \chapterpageref{fenum-checker})
\item
  \ahrefloc{tainting-checker}{Tainting Checker} for trust and security errors
  (see \chapterpageref{tainting-checker})
\item
  \ahrefloc{guieffect-checker}{GUI Effect Checker} to ensure that non-GUI
  threads do not access the UI, which would crash the application
  (see \chapterpageref{guieffect-checker})
\item
  \ahrefloc{units-checker}{Units Checker} to ensure operations are
  performed on correct units of measurement
  (see \chapterpageref{units-checker})
\item
  \ahrefloc{aliasing-checker}{Aliasing Checker} to identify whether
  expressions have aliases (see \chapterpageref{aliasing-checker})
\item
  \ahrefloc{linear-checker}{Linear Checker} to control aliasing and prevent
  re-use (see \chapterpageref{linear-checker})
\item
  \ahrefloc{igj-checker}{IGJ Checker} for mutation errors (incorrect
  side effects), based on the IGJ type system (see
  \chapterpageref{igj-checker})
\item
  \ahrefloc{javari-checker}{Javari Checker} for mutation errors
  (incorrect side effects), based on the Javari type system (see
  \chapterpageref{javari-checker})
\item
  \ahrefloc{subtyping-checker}{Subtyping Checker} for customized checking without
  writing any code (see \chapterpageref{subtyping-checker})
\end{itemize}



\subsection{Fields and flow-sensitive analysis\label{type-refinement-fields}}

Flow sensitivity analysis infers the type of fields in some restricted cases:

\begin{itemize}

\item
A final initialized field:
Type inference is performed for final fields that are initialized to a
compile-time constant at the declaration site; so the type of \code{protocol}
is \code{@NonNull String} in the following declaration:

\begin{Verbatim}
    public final String protocol = "https";
\end{Verbatim}

Please note that such inferred type may leak to the public interface of the
class.  To override such behavior, you can explicitly insert the desired
annotation, e.g.,

\begin{Verbatim}
    public final @Nullable String protocol = "https";
\end{Verbatim}

\item
Within method bodies:
Type inference is performed for fields in the context of method bodies,
like local variables, but method invocations invalidate any inferred
information.  Consider the following example, where \code{updatedAt} is a nullable
field:

\begin{Verbatim}
class DBObject {
  @Nullable Date updatedAt;

  void persistData() {
    ... // write to disk or other non-volatile memory
    updatedAt = null;
  }

  void update() {
    if (updatedAt == null)
        updatedAt = new Date();
    // updatedAt is nonnull
    log("Updating object at " + updatedAt.getTime());

    persistData();
    // updatedAt is nullable again
    log.debug("Saved object updated at " + updatedAt.getTime()); // invalid!
  }
}
\end{Verbatim}

Here the call to \code{persistData()} invalidates the inferred non-null type
of \code{updatedAt}.

When methods do not modify any object state or have any identity side
effects (e.g., \code{log()} method here), you can annotate these methods as
\code{SideEffectFree} or \code{Pure} (see
Section~\ref{type-refinement-purity}).  When a method is annotated as
\code{SideEffectFree}, the flow analyzer carries the inferred types across
the method invocation boundary.

\end{itemize}


\subsection{Side effects, determinism, purity, and flow-sensitive analysis\label{type-refinement-purity}}

As described above, a checker can use a refined type for an expression from
the time when the checker infers that the value has that refined type,
until the checker can no longer support that inference.

The refined type begins at a test (such as \<if (myvar != null) ...>) or
an assignment.  If the assignment occurs within a method body,
write a postcondition annotation such as
\refqualclass{checker/nullness/qual}{EnsuresNonNull}.

The refined type ends at an assignment or possible assignment.  Any method
call has the potential to side-effect any field, so calling a method
typically causes the checker to discard its knowledge of the refined type.
This is undesirable if the method doesn't actually re-assign the field.

There are three annotations, collectively called purity annotations, that
you can use to help express what effects a method call does not have.
Usually, you only need to use \refqualclass{dataflow/qual}{SideEffectFree}.


\begin{description}

\item[\refqualclass{dataflow/qual}{SideEffectFree}]
  indicates that the method has no (visible) side effects.

\item[\refqualclass{dataflow/qual}{Deterministic}]
  indicates that if the method is called multiple times with the same
  arguments, then it returns the same result.

\item[\refqualclass{dataflow/qual}{Pure}]
  indicates that the method is both @SideEffectFree and @Deterministic.

\end{description}

The Javadoc of the annotations describes their semantics and how they are
checked.  This manual section gives examples and supplementary information.

For example, consider the
following declarations and uses:

\begin{Verbatim}
        @Nullable Object myField;

        int computeValue() { ... }

        ...
        if (myField != null) {
          int result = computeValue();
          myField.toString();
        }
\end{Verbatim}

\noindent
Ordinarily, the Nullness Checker would issue a warning regarding the
\code{toString()} call, because the receiver \code{myField} might be
\code{null}, according to the \code{@Nullable} annotation on the
declaration of \code{myField}.  Even though the code checked the value of
\code{myField}, the call to \<computeValue> might have re-set it to null.
If you change the declaration of \code{computeValue} to

\begin{Verbatim}
        @SideEffectFree int computeValue() { ... }
\end{Verbatim}

\noindent
then the Nullness Checker issues no warnings, because it can reason that
the second occurrence of \code{myField} has the same (non-null) value as
the one in the test.

As a more complex example, consider the
following declaration and uses:

\begin{Verbatim}
        @Nullable Object getField(Object arg) { ... }

        ...
        if (x.getField(y) != null) {
          x.getField(y).toString();
        }
\end{Verbatim}

Ordinarily, the Nullness Checker would issue a warning regarding the
\code{toString()} call, because the receiver \code{x.getField(y)} might
be \code{null}, according to the \code{@Nullable} annotation in the
declaration of \code{getField}.  If you change the declaration of
\code{getField} to

\begin{Verbatim}
        @Pure @Nullable Object getField(Object arg) { ... }
\end{Verbatim}

\noindent
then the Nullness Checker issues no warnings, because it can reason that
the two invocations \code{x.getField(y)} have the same value, and
therefore that \code{x.getField(y)} is non-null within the then branch
of the if statement.


If a method is side-effect-free or pure, then it would be legal to annotate its receiver and
every parameter as \<@ReadOnly>, in the IGJ (Chapter~\ref{igj-checker}) or
Javari (Chapter~\ref{javari-checker}) type systems.  The reverse is not
true, because the method might side-effect a global variable.  (Also, for
the case of \<@Pure>, the method might not be deterministic.)

If you supply the command-line option \<-AsuggestPureMethods>, then the
Checker Framework will suggest methods that can be marked as
\<@SideEffectFree>, \<@Deterministic>, or \<@Pure>.

% Maybe this ambiguous option, which enables checking whether method
% bodies are compatible with the method declaration annotation, should be
% renamed, such as to -AcheckPurity.

Currently, purity annotations are trusted.  Purity annotations on called
methods affect type-checking of client code.  However, you can make a
mistake by writing \<@SideEffectFree> on the declaration of a method that
is not actually side-effect-free or by writing \<@Deterministic> on the
declaration of a method that is not actually deterministic.  To enable
checking of the annotations, supply the command-line option
\<-AcheckPurityAnnotations>.  It is not enabled by default because of a high false
positive rate.  In the future, after a new purity-checking analysis is
implemented, the Checker Framework will default to checking purity
annotations.

It can be tedious to annotate library methods with purity annotations such
as \<@SideEffectFree>.  If you supply the command-line option
\<-AassumeSideEffectFree>, then the Checker Framework will unsoundly
assume that every called method is side-effect-free.  This can make
flow-sensitive type refinement much more effective, since method calls will
not cause the analysis to discard information that it has learned.
However, this option can mask real errors.  It is most appropriate when you
are starting out annotating a project, or if you are using the Checker
Framework to find some bugs but not to give a guarantee that no more errors
exist of the given type.

A common error is:

\begin{Verbatim}
MyClass.java:1465: error: int hashCode() in MyClass cannot override int hashCode(Object this) in java.lang.Object; attempting to use an incompatible purity declaration
    public int hashCode() {
               ^
  found   : []
  required: [SIDE_EFFECT_FREE, DETERMINISTIC]
\end{Verbatim}

\noindent
The reason for the error is that the \<Object> class is annotated as:

\begin{Verbatim}
class Object {
  ...
  @Pure int hashCode() { ... }
}
\end{Verbatim}

\noindent
(where \refqualclass{dataflow/qual}{Pure} means both
\refqualclass{dataflow/qual}{SideEffectFree} and
\refqualclass{dataflow/qual}{Deterministic}).  Every overriding
definition, including those in your program, must use be at least as strong
a specification; in particular, every overriding definition must be
annotated as \<@Pure>.

You can fix the definition by adding \<@Pure> to your method definition.
Alternately, you can suppress the warning.
You can suppress each such warning individually using
\<@SuppressWarnings("purity.invalid.overriding")>,
or you can use the \<-AsuppressWarnings=purity.invalid.overriding>
command-line argument to suppress all such warnings.
% The \<-AassumeSideEffectFree> command-line argument suppresses the
% warning, but it does more too; don't mention it here.
In the future, the Checker Framework will support inheriting annotations
from superclass definitions.



The \refqualclass{dataflow/qual}{TerminatesExecution} annotation
indicates that a given method never returns.  This can enable the
flow-sensitive type refinement to be more precise.


% This is a slightly funny place for this section, but I cannot decide on a
% better one.
\subsection{Assertions\label{type-refinement-assertions}}

If your code contains an \<assert> statement, then your code could behave
in two different ways at run time, depending on whether assertions are
enabled or disabled
via the \<-ea> or \<-da> command-line options to java.

By default, the Checker Framework outputs warnings about any error that
could happen at run time, whether assertions are enabled or disabled.

If you supply the \<-AassumeAssertionsAreEnabled> command-line option, then
the Checker Framework assumes assertions are enabled.  If you supply the
\<-AassumeAssertionsAreDisabled> command-line option, then the Checker
Framework assumes assertions are disabled.  You may not supply both
command-line options.  It is uncommon to supply either one.

These command-line arguments have no effect on processing of \<assert>
statements whose message contains the text \<@AssumeAssertion>; see
Section~\ref{assumeassertion}.


% If you add a Javadoc link to this location, also add the qualifier to the
% list below.
\section{Writing Java expressions as annotation arguments\label{java-expressions-as-arguments}}

Sometimes, it is necessary to write a Java expression as the argument to an
annotation.  The annotations that take a Java
expression as an argument include:

% TODO: Need to periodically check/update this list.
\begin{itemize}
\item \refqualclass{framework/qual}{RequiresQualifier}
\item \refqualclass{framework/qual}{EnsuresQualifier}
\item \refqualclass{framework/qual}{EnsuresQualifierIf}
\item \refqualclass{checker/nullness/qual}{RequiresNonNull}
\item \refqualclass{checker/nullness/qual}{EnsuresNonNull}
\item \refqualclass{checker/nullness/qual}{EnsuresNonNullIf}
% Not implemented: \refqualclass{checker/nullness/qual}{AssertNonNullIfNonNull}
\item \refqualclass{checker/nullness/qual}{KeyFor}
\item \refqualclass{checker/i18nformatter/qual}{I18nFormatFor}
\item \refqualclass{checker/lock/qual}{EnsuresLockHeld}
\item \refqualclass{checker/lock/qual}{EnsuresLockHeldIf}
\item \refqualclass{checker/lock/qual}{GuardedBy}
\item \refqualclass{checker/lock/qual}{Holding}
\end{itemize}

The set of permitted expressions is a subset of all Java expressions:

\begin{itemize}
\item
  the receiver object, \<this>.  You can write \<this> to annotate any
  variable or declaration where you could write \<this> in code.
  Notably, it cannot be used in annotations on declarations of
  static fields or methods.  For a field, \<this> is the field's
  receiver, i.e. its container.  For a local variable, it is the
  method's receiver.
  Note that assignment of type qualifier annotations with argument
  \<this> is conservatively disallowed because the receiver may fall
  out of scope.  For example, suppose the annotation on \<obj2.key>
  has been refined to \<@KeyFor("this")>.  Then the assignment
  \code{obj1.key = obj2.key} is disallowed, because \<obj2> may
  fall out of scope while \<obj1> is still in use.

\item
  the receiver object as seen from the superclass, \<super>.  This can be used
  to refer to fields shadowed in the subclass (although shadowing fields is
  discouraged in Java).

\item
  \<itself>, i.e. the value of the annotated reference (non-primitive) variable.
  Currently only defined for the \<@GuardedBy> type system.
  For example, \<@GuardedBy("itself") Object o> indicates that the value
  referenced by \<o> is guarded by the intrinsic (monitor) lock of the value
  referenced by \<o>.

\item
  a formal parameter.  Write \<\#> followed by the \textbf{one-based} parameter
  index.  For example: \<\#1>, \<\#3>.  It is not permitted to write \<\#0> to
  refer to the receiver object; use \<this> instead.  (A side note:
  The formal parameter syntax \<\#1> is less natural in source code
  than writing the formal parameter name.  This syntax is necessary for
  separate compilation, when an annotated method has already been compiled
  into a \<.class> file and a client of that method is later compiled.
  In the \<.class> file, no formal parameter name information is available,
  so it is necessary to use a number to indicate a formal parameter.)
%% At some point, we will start assuming that the .class file was created
%% by the Checker Framework and this justification, and the #n syntax
%% itself, will not be necessary.
% Running
% the compiler without a checker should create legal annotations in the
% \<.class> file, so we cannot rely on the checker to translate names to
% indices.

% This is not implemented at the moment, as KeyFor does not use flow expressions yet.
%\item
%  a local variable.  This is not applicable for method annotations, but is
%  applicable to type annotations such as
%  \refqualclass{checker/nullness/qual}{KeyFor}.  Write the variable name.  For
%  example: \<myLocalVar>.
\item
  a static variable.  Write the class name and the variable, as in
  \<System.out>.
  % It may also work to use an instance variable and the static variable
  % name, but I'm not sure about that.

\item
  a field of any expression.  For example:  \<next>,
  \<this.next>, \<\#1.next>. %, \<myLocalVar.next>.
  You may optionally omit a leading ``\<this.>'', just as in Java.  Thus,
  \<this.next> and \<next> are equivalent.

\item
  an array access.  For example:  \<this.myArray[i]>, \<vals[\#1]>.

\item literals: string, integer, long, null.

\item a method invocation on any expression.
  This even works for overloaded methods and methods with type parameters.
  For example:
  \<m1(x, y.z, \#2)>, \<a.m2("hello")>.
  One unusual feature is that the method call is allowed to have side
  effects.  If a specification is going to be checked at run time via
  assertions, then the specification must not use methods with side
  effects.  But, the Checker Framework works at compile time, so it allows
  side effects.
  The current implementation will never able to prove such
  a contract, but it is able to use the information (when checking
  the method body with preconditions, or when checking the caller's
  code with postconditions).  This can be useful to annotate trusted
  methods precisely (e.g., \<java.io.BufferedReader.ready()>).  % TODO: expand this example.
\end{itemize}

% We want this in the future:
% The expression may refer to private fields and method calls.  The client
% cannot directly make use of these private fields and method calls, but they
% may be useful in establishing a precondition that itself refers to private
% fields or methods.  This exposure of implementation details is unfortunate,
% but it enables more expressive and useful specifications to be written, it does not jeopardize soundness nor let the client manipulate the
% representation, and it is simple
% simpler than defining specification fields or ghost fields
%% TODO: we may eventually have an @SpecField annotation


\textbf{Limitations:}
The following Java expressions may not currently be written:
\begin{itemize}
\item Some literals:  floats, doubles, chars, and class literals.
\item String concatenation expressions.
\item Mathematical operators (plus, minus, division, ...).
\item Comparisons (equality, less than, etc.).
\item Quantification over all array components (e.g. to express that all
  array elements are non-null).
\end{itemize}



% \section{Unused fields and dependent types\label{unused-fields-and-dependent-types}}
\section{Unused fields\label{unused-fields}}

In an inheritance hierarchy, subclasses often introduce new methods and
fields.  For example, a \<Marsupial> (and its subclasses such as
\<Kangaroo>) might have a variable \<pouchSize> indicating the size of the animal's
pouch.  The field does not exist in superclasses such as
\<Mammal> and \<Animal>, so Java issues a compile-time
error if a program tries to access \<myMammal.pouchSize>.

If you cannot use subtypes in your program, you can enforce similar
requirements using type qualifiers.
For fields, use the \<@Unused> annotation (Section~\ref{unused-annotation}), which enforces that a field or method may only
be accessed from a receiver expression with a given annotation (or one of
its subtypes).
For methods, annotate the receiver parameter \<this>; then a method call
type-checks only if the actual receiver is of the specified type.

% For example, consider the declaration
%   void getChars(@Untainted String this, int srcBegin, int srcEnd, char[] dst,
%   int dstBegin) { ... }
% Now, the invocation
%   myString.getChars(a, b, c, d)
% type-checks only if myString has type @Untainted String.  It does not
% type-check if myString has type @Tainted String.


% Then,
% Section~\ref{dependent-types} describes an even more powerful mechanism, by
% which the qualifier of a field depends on the qualifier of the expression
% from which the field was accessed.

Also see the discussion of typestate checkers, in
Chapter~\ref{typestate-checker}.


% \subsection{Unused fields\label{unused-fields}}
\subsection{\<@Unused> annotation\label{unused-annotation}}

A Java subtype can have more fields than its supertype.  For example:

\begin{Verbatim}
class Animal { }
class Mammal extends Animal { ... }
class Marsupial extends Mammal {
  int pouchSize;  // pouch capacity, in cubic centimeters
  ...
}
\end{Verbatim}

You can simulate
the same effect for type qualifiers:
the \refqualclass{framework/qual}{Unused} annotation
on a field declares that the field may \emph{not} be accessed via a receiver of
the given qualified type (or any \emph{super}type).
% a given field \emph{may not} be accessed via
% a reference with a supertype qualifier, but \emph{may} be accessed via a reference
% with a subtype qualifier.
%% The following is true, but there's no need to distract readers with this detail.
% (It would probably be clearer to replace \<@Unused> by an annotation that
% indicates when the field \emph{may} be used.)
For example:

\begin{Verbatim}
class Animal {
  @Unused(when=Mammal.class)
  int pouchSize;  // pouch capacity, in cubic centimeters
  ...
}
@interface Mammal { }
@interface Marsupial { }

@Marsupial Animal joey = ...;
... joey.pouchSize ...    // OK
@Mammal Animal mae = ...;
... mae.pouchSize ...    // compile-time error
\end{Verbatim}

The above class declaration is like writing

\begin{Verbatim}
class @Mammal-Animal { ... }
class @Marsupial-Animal {
  int pouchSize;  // pouch capacity, in cubic centimeters
  ...
}
\end{Verbatim}


% \subsection{Dependent types\label{dependent-types}}
%
% A variable has a \emph{dependent type} if its type depends on some other
% value or type.
% %  --- the type is dynamically, not statically, determined.
% % (Type-safety can still be statically determined, though.)
%
% The Checker Framework supports a form of dependent types, via the
% \refqualclass{framework/qual}{Dependent} annotation.
% This annotation changes the type of a reference, based on the
% qualified type of the receiver (\code{this}).  This can be viewed as a more
% expressive form of polymorphism (see Section~\ref{polymorphism}).  It can
% also be seen as a way of linking the meanings of two type qualifier
% hierarchies.
%
% When the \refqualclass{framework/qual}{Unused} annotation is sufficient, you
% should use it instead of \code{@Dependent}.
%
% Here is a restatement of the example of Section~\ref{unused-fields}, using
% \refqualclass{framework/qual}{Dependent}:
%
% \begin{Verbatim}
% @interface Mammal { }
% @interface Marsupial { }
% class Animal {
%   // pouch capacity, in cubic centimeters
%   // (non-null if this animal is a marsupial)
%   @Nullable @Dependent(result=NonNull.class, when=Marsupial.class) Integer getPouchSize;
%   ...
% }
%
% @Marsupial Animal joey = ...;
% ... joey.getPouchSize().intValue() ...    // OK
% @Mammal Animal mae = ...;
% ... mae.getPouchSize().intValue() ...    // compile-time error:
%                                          //   dereference of possibly-null mae.getPouchSize()
% \end{Verbatim}
%
% Just as writing \<@Unused> is similar to writing multiple classes (but when
% it is not possible to write real subclasses), \<@Dependent> mimics the
% effect of multiple classes with overriding definitions of some method or
% field.
% The above class declaration is like writing
%
% \begin{Verbatim}
% class @Mammal-Animal {
%   // pouch capacity, in cubic centimeters
%   // (non-null if this animal is a marsupial)
%   @Nullable Integer getPouchSize();
% }
% class @Marsupial-Animal {
%   // pouch capacity, in cubic centimeters
%   @NonNull Integer getPouchSize();
%   ...
% }
% \end{Verbatim}
%
%
% \subsubsection{Limitations of \<@Dependent>\label{dependent-types-limitations}}
%
% It is unsound to write \<@Dependent> on a non-\<final> field.  Consider the
% following:
%
% \begin{Verbatim}
% class MyClass {
%   @Nullable @Dependent(result=NonNull.class, when=Marsupial.class) Integer pouchSize;
% }
% \end{Verbatim}
%
% Then it would be possible to write
%
% \begin{Verbatim}
% @Marsupial Animal a = new @Marsupial Animal();
% @Mammal Animal b = a;
% b.pouchSize = null;
% a.pouchSize.intValue();
% \end{Verbatim}
%
% \noindent
% In the last line of the example, \<a.pouchSize> is null, contradicting its
% declaration and leading to a null pointer exception that would not be
% caught by the type-checker.
%
% In certain circumstances, it may be desirable to write such an annotation
% on a \<private> field anyway, manually check all uses, and then suppress
% the warning.
%
% It is sound to write \<@Dependent> on a \<final> field.  Such a field
% behaves like a getter method, such as \<getPouchSize()> above.



% TO DO:  give an example where @Dependent is actually needed


% LocalWords:  MyClass qual PolymorphicQualifier DefaultQualifier subpackages
% LocalWords:  DefaultQualifiers actuals toArray CollectionToArrayHeuristics nn
% LocalWords:  MyList Nullness DefaultLocation nullness PolyNull util java TODO
% LocalWords:  QualifierDefaults nullable lub persistData updatedAt nble KeyFor
% LocalWords:  subtype's RequiresNonNull EnsuresNonNull EnsuresNonNullIf
% LocalWords:  myLocalVar myClass getPackage getSuperclass myString Regex
% LocalWords:  getComponentType enum implementers dereferenced superclasses
%  LocalWords:  regex myStrings myVar pouchSize myMammal getter foo MyAnno
%  LocalWords:  getPouchSize TerminatesExecution myvar myField getField m1
%  LocalWords:  computeValue ReadOnly IGJ AsuggestPureMethods Instanceof
%  LocalWords:  AcheckPurityAnnotations AassumeSideEffectFree iMplicit instanceof m2
%  LocalWords:  AassumeAssertionsAreEnabled myArray vals propkey forName
%  LocalWords:  fenum igj javari i18n RequiresQualifier EnsuresQualifier
%  LocalWords:  EnsuresQualifierIf AsuppressWarnings AinvariantArrays asts
%  LocalWords:  formatter ocals nstanceof plicit ounds wildcard's da
%%  LocalWords:  wildcards guieffect AassumeAssertionsAreDisabled
%%  LocalWords:  AssumeAssertion AsafeDefaultsForUnannotatedBytecode
%%  LocalWords:  I18nFormatFor AunsafeDefaultsForUnannotatedBytecode
