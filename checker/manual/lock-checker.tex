\htmlhr
\chapter{Lock Checker\label{lock-checker}}

The Lock Checker prevents certain concurrency errors by enforcing a
locking discipline.  A locking discipline indicates which locks must be held
when a given operation occurs.  You express the locking discipline by
giving a variable a type of
\<\refqualclass{checker/lock/qual}{GuardedBy}{\small("\emph{lockexpr}")}>.
This indicates that the variable may
be dereferenced only if the given lock is held.


To run the Lock Checker, supply the
\code{-processor org.checkerframework.checker.lock.LockChecker}
command-line option to javac.


\section{What the Lock Checker guarantees\label{lock-guarantees}}

The Lock Checker gives the following guarantee.
Suppose that expression $e$ has type
\<\refqualclass{checker/lock/qual}{GuardedBy}(\ttlcb"x", "y.z"\ttrcb)>.
Then the value computed for $e$ is only dereferenced by a thread when the
thread holds locks \<x> and \<y.z>.
% This seems obvious
% at the time of the dereference.
Dereferencing a value is reading or writing one of its fields, or
calling a method with it as the receiver.
The guarantee holds regardless of whether the dereference uses the expression $e$
directly, or the value was first copied into a variable, returned as the
result of a method call, etc.
Copying a reference is always
permitted by the Lock Checker, regardless of which locks are held.

A lock is held if it has been acquired but not yet released.
Java has two types of locks.
A monitor lock is acquired upon entry to a \<synchronized> method or block,
and is released on exit from that method or block.
%  (More precisely,
%  the current thread locks the monitor associated with the value of
%  \emph{E}; see \href{https://docs.oracle.com/javase/specs/jls/se8/html/jls-17.html#jls-17.1}{JLS \S17.1}.)
An explicit lock is acquired by a method call such as
\sunjavadoc{java/util/concurrent/locks/Lock.html\#lock()}{Lock.lock()},
and is released by another method call such as
\sunjavadoc{java/util/concurrent/locks/Lock.html\#unlock()}{Lock.unlock()}.
The Lock Checker enforces that any expression whose type implements
\sunjavadoc{java/util/concurrent/locks/Lock.html}{Lock} is used as an
explicit lock, and all other expressions are used as monitor locks.

Ensuring that your program obeys its locking discipline is an easy and
effective way to eliminate a common and important class of errors.
If the Lock Checker issues no warnings, then your program obeys its locking discipline.
However, your program might still have other types of concurrency errors.
For example, you might have forgotten to indicate that a particular value
should only be dereferenced when a lock is held.  Your program might release and
re-acquire the lock, when correctness requires it to hold it throughout a
computation.  And, there are other concurrency errors that cannot, or
should not, be solved with locks.


\section{Lock annotations\label{lock-annotations}}

This section describes the lock annotations you can use on types and methods.


\subsection{Type qualifiers\label{lock-type-qualifiers}}

\begin{description}

\item[\refqualclass{checker/lock/qual}{GuardedBy}{\small(\emph{exprSet})}]
  If a variable \<x> has type \<@GuardedBy("\emph{expr}")>, then a thread may
  dereference the value referred to by \<x> only when the thread holds the
  lock that \emph{expr} currently evaluates to.

  It is also permitted to write multiple expressions as in
  \<@GuardedBy(\ttlcb"\emph{expr1}", "\emph{expr2}"\ttrcb)>, in which case
  the dereference is
  permitted only if the thread holds all the locks.

  \<@GuardedBy(\{\})> is the default type qualifier if the programmer does not
  write an explicit locking type qualifier.

% Description adapted from that of @NonLeaked in the Aliasing Checker chapter.
% @NonLeakedUnknownGuard is inspired from @NonLeaked but its meaning is different.
\item[\refqualclass{checker/lock/qual}{NonLeakedUnknownGuard}]
  identifies a formal parameter that is not leaked nor
  returned by the method body, and for which the \code{@GuardedBy} annotation
  on the actual parameter is unknown at the method definition site.
  For example, the formal parameter of the String copy constructor,
  \code{String(String s)}, is \code{@NonLeakedUnknownGuard} because the body of the method
  only makes a copy of the parameter, and the @GuardedBy annotation
  on the actual parameter to a call to the constructor is unknown
  at its definition site, because the constructor can be called by
  arbitrary code.  See Section~\ref{lock-checker-library-methods}
  for details on when to use \code{@NonLeakedUnknownGuard}.

  We say a routine "leaks" a value if the routine makes the value accessible
  to clients, by making it reachable from a value accessible to the client
  such as a global variable, an argument, a field, or a return value.

% Description adapted from that of @LeakedToResult in the Aliasing Checker chapter.
% @LeakedToResultUnknownGuard is inspired from @LeakedToResult but its meaning is different.
\item[\refqualclass{checker/lock/qual}{LeakedToResultUnknownGuard}]
  is used when the parameter may be returned, but it is not
  otherwise leaked, and for which the @GuardedBy annotation
  on the actual parameter is unknown at the method definition site.
  For example, the receiver parameter of \code{StringBuffer.append(StringBuffer
  this, String s)} is
  \code{@LeakedToResultUnknownGuard}, because the method returns the updated receiver,
  and the @GuardedBy annotation
  on the receiver of a call to the method is unknown
  at its definition site, because the method can be called by
  arbitrary code.  See Section~\ref{lock-checker-library-methods}
  for details on when to use \code{@LeakedToResultUnknownGuard}.

\item[\refqualclass{checker/lock/qual}{GuardedByInaccessible}]
  If a variable \<x> has type @GuardedByInaccessible, then
  the value referred to by the \<x> can never be accessed.
  This annotation is used internally by the type system
  and should never be written by a programmer.

\item[\refqualclass{checker/lock/qual}{GuardedByBottom}]
  This annotation is used internally by the type system
  and should never be written by a programmer.

\end{description}

\begin{figure}
\includeimage{lock-guardedby}{2.35cm}
\caption{The subtyping relationship of the Lock Checker's qualifiers.
\code{@GuardedBy(\{\})} is the default type qualifier for all unannotated
variables.
Qualifiers in gray
are used internally by the type system but should never be written by a
programmer.}
\label{fig-lock-guardedby-hierarchy}
\end{figure}

Figure~\ref{fig-lock-guardedby-hierarchy} shows the type hierarchy of these
qualifiers.
All \code{@GuardedBy} annotations are incomparable:
if \emph{Eset1} $\neq$ \emph{Eset2}, then \code{@GuardedBy(\emph{Eset1})} and
\code{@GuardedBy(\emph{Eset2})} are siblings in the type hierarchy.
It might be expected that
\<@GuardedBy({"x", "y"}) T> is a supertype of \<@GuardedBy({"x"}) T>.  The
first type requires two locks to be held, and the second requires only one
lock to be held and so could be used in any situation where both locks are
held.  The type system conservatively prohibits this in order to prevent
type-checking loopholes that would result from aliasing and side effects
--- that is, from having two mutable references, of different types, to the
same data. See
Section~\ref{lock-examples-guardedby-and-holding} for an example
of a problem that would occur if this rule were relaxed.


\subsection{Declaration annotations\label{lock-declaration-annotations}}

The Lock Checker supports several annotations that specify method behavior.
These are declaration annotations, not type annotations: they apply to the
method itself rather than to some particular type.

% If changing this text, make sure it fits in the page in the PDF.
% The wording below was chosen to ensure that it fits.
\begin{tabular}{l|l|}
\textbf{Declaration annotation} & \textbf{Meaning} \\
@Holding(String[] locks) &
Lock expressions that must be held when the method is called.
\\
@EnsuresLockHeld(String[] locks) &
Lock expressions guaranteed to be held on method return.
\\
@EnsuresLockHeldIf(String[] locks, boolean result) &
Lock expressions guaranteed to be held when the given result is returned.
\\
@LockingFree &
The method does not make any use of locks or synchronization.
\\
\end{tabular}


\begin{description}
\item[\refqualclass{checker/lock/qual}{Holding}\small{(String[] locks)}]
  A method invocation type-checks only if all the lock expressions
  mentioned in a \<@Holding> precondition are held at the method call site.
\item[\refqualclass{checker/lock/qual}{EnsuresLockHeld}\small{(String[] locks)}]
\item[\refqualclass{checker/lock/qual}{EnsuresLockHeldIf}\small{(String[] locks, boolean result)}]
  indicate a method postcondition.  With \code{@EnsuresLockHeld}, the given
  expressions are known by the programmer to be in
  a locked state after the method returns; this is useful for annotating a
  method that acquires a lock such as
  \sunjavadoc{java/util/concurrent/locks/ReentrantLock.html\#lock()}{ReentrantLock.lock()}.

  With \code{@EnsuresLockHeldIf}, if the annotated method returns the given
  boolean value (true or false), the given
  expressions are known by the programmer to be in
  a locked state after the method returns; this is useful for annotating a
  method that conditionally acquires a lock.
  See Section~\ref{ensureslockheld-examples} for examples.

\item[\refqualclass{dataflow/qual}{LockingFree}]
  indicates that the method does not use synchronization/locking,
  directly or indirectly.  This is less restrictive than
  \refqualclass{dataflow/qual}{SideEffectFree}. It is especially useful for
  annotating library methods, including JDK methods. Since
  \code{@SideEffectFree} implies \code{@LockingFree}, if both are applicable
  then you should only write \code{@SideEffectFree}.
  Do not use this annotation on a method that acquires locks but then releases
  them, leaving the same number of locks held on exit as on entry.  Such a
  method is not locking-free and may cause deadlocks.  If you need to annotate
  such a method when type-checking with the Lock Checker, consult the
  Checker Framework developers for assistance.

\textcolor{red}{MDE: Can we say something more specific here? What should a developer do?}

\textcolor{red}{JT: How about we add a new @LockCountNeutral annotation
to indicate that a method releases all locks it acquires, and is otherwise @LockingFree?
For now, the Lock Checker would treat it as @LockingFree.  Later, if we add a deadlock
detection system, we could ask users interested
in deadlock avoidance to replace this annotation with a more specific one (for example, one
that specifies all the locks that are acquired/released by the method).}

\end{description}


\subsection{Discussion of \<@Holding>\label{lock-checker-holding}}

A programmer might choose to use the \code{@Holding} method annotation in
two different ways:  to specify a higher-level protocol, or to summarize
intended usage.  Both of these approaches are useful, and the Lock Checker
supports both.

\paragraph{Higher-level synchronization protocol\label{lock-checker-holding-highlevel}}

\textcolor{red}{MDE: Can you give a concrete example? This is abstract, and it seems
to say you will use locks for protocols that are not expressible as locks, which is confusing.}

\textcolor{red}{JT: Sorry I forgot to re-examine and re-write this paragraph. This paragraph
was added in this form in 2009. I believe it might be related to the notion that the Lock
Checker can be instructed to treat an arbitrary class as if it were a lock class. For example,
the Monitor class in Guava can be annotated such that the Lock Checker will interpret
@Holding("monitor") correctly, making it unnecessary to explicitly write @Holding("monitor.lock").}

\textcolor{red}{Do you think that might be related to the meaning of this paragraph? If not, I propose we remove
it (and the reference to it right above).}

  \code{@Holding} can specify a higher-level synchronization protocol that
  is not expressible as locks over Java objects.  By requiring locks to be
  held, you can create higher-level protocol primitives without giving up
  the benefits of the annotations and checking of them.

\paragraph{Method summary that simplifies reasoning\label{lock-checker-holding-method-summary}}

  \code{@Holding} can be a method summary that simplifies reasoning.  In
  this case, the \code{@Holding} doesn't necessarily introduce a new
  correctness constraint; the program might be correct even if the lock
  were acquired later in the body of the method or in a method it calls, so
  long as the lock is acquired before dereferencing the value it protects.

  Rather, here \code{@Holding} expresses a fact about execution:  when
  execution reaches this point, the following locks are already held.  This
  fact enables people and tools to reason intra- rather than
  inter-procedurally.

  In Java, it is always legal to re-acquire a lock that is already held,
  and the re-acquisition always works.  Thus, whenever you write

\begin{Verbatim}
  @Holding("myLock")
  void myMethod() {
    ...
  }
\end{Verbatim}

\noindent
it would be equivalent, from the point of view of which locks are held
during the body, to write

\begin{Verbatim}
  void myMethod() {
    synchronized (myLock) {   // no-op:  re-acquire a lock that is already held
      ...
    }
  }
\end{Verbatim}


It is better to write a \code{@Holding} annotation rather than writing the
extra synchronized block.  Here are reasons:

\begin{itemize}
\item
  The annotation documents the fact that the lock is intended to already be
  held;  that is, the method's contract requires that the lock be held when
  the method is called.
\item
  The Lock Checker enforces that the lock is held when the method is
  called, rather than masking a programmer error by silently re-acquiring
  the lock.
\item
  The synchronized statement can deadlock if, due to a programmer error,
  the lock is not already held.  The Lock Checker prevents this type of
  error.
\item
  The annotation has no run-time overhead.  Even if the lock re-acquisition
  succeeds, it still consumes time.
\end{itemize}


\subsection{Annotating library methods\label{lock-checker-library-methods}}

\paragraph{Using the \refqualclass{checker/lock/qual}{NonLeakedUnknownGuard} annotation\label{lock-checker-library-methods-nonleakedunknownguard}}

Often, when annotating library methods, one encounters methods that can
be annotated with \code{@LockingFree}.  However, in order to make calls
to the library methods type-check without warnings, a \code{@LockingFree}
annotation is often not sufficient.  Consider the following JDK method declaration:

\begin{verbatim}
public String(String original)
\end{verbatim}

The documentation for \sunjavadoc{java/lang/String.html\#String-java.lang.String-}{String(String original)}
indicates that the newly created string is a copy of the original string.
It is clear that this constructor's behavior does not make any use of
locks or synchronization, therefore it is appropriate to annotate it
with \code{@LockingFree}:

\begin{verbatim}
@LockingFree
public String(String original)
\end{verbatim}

Now consider the following user code that calls this method:

\begin{verbatim}
ReentrantLock lock;
@GuardedBy("lock") String filename;
...
lock.lock();
String copyOfFilename = String(filename);
filename = filename.concat(".txt");
\end{verbatim}

The \code{@LockingFree} annotation is useful because it has indicated to the
Lock Checker that all the lock expressions that were held prior to the
call to \code{String(filename)} are still held after the call, and so the
last line type-checks without warnings.  However a warning is still issued
for the line that calls \code{String(filename)} because the value of
\code{filename} is guarded by \code{lock}, whereas the formal parameter
\code{String original} is \code{@GuardedBy(\{\})}.

The solution cannot be to change the declaration of the constructor to
indicate that the formal parameter is \code{@GuardedBy("lock")}, since
\code{lock} is undefined in the context of the declaration, and furthermore
this would require all callers of the constructor to guard the string
using a variable named \code{lock}, which in addition to being impractical
is ambiguous as \code{lock} could refer to different variables in different
calling contexts.

Instead, the recommended solution is to annotate the declaration as follows:

\begin{verbatim}
@LockingFree
public String(@NonLeakedUnknownGuard String original)
\end{verbatim}

Because the documentation of the constructor indicated that a copy
is made of the original string, it is known that a reference to the
original string will not be leaked (except possibly to local variables
used within the method body).

With this \code{@NonLeakedUnknownGuard} annotation in place, the
Lock Checker now issues no warnings for the \code{String(filename)}
constructor call in the user code.  This is because (1) \code{lock}
is known to be held prior to the call to the constructor, (2) the
value in \code{filename} is known not to be leaked except to local
variables , and (3) the method is annotated with \code{@LockingFree},
therefore it is not possible for \code{lock} to be unlocked in the
middle of the method execution before \code{filename} is dereferenced
by the method.  The entire user sample code now type-checks without
warnings.

\paragraph{Using the \refqualclass{checker/lock/qual}{LeakedToResultUnknownGuard} annotation\label{lock-checker-library-methods-leakedtoresultunknownguard}}

Consider now the following library method declaration:

\begin{verbatim}
@LockingFree
public StringBuffer append(StringBuffer this, String str)
\end{verbatim}

and the following user code that calls it:

\begin{verbatim}
ReentrantLock lock;
@GuardedBy("lock") StringBuffer filename;
...
lock.lock();
filename = filename.append(".txt");
\end{verbatim}
% Technically, the 'filename = ' is unnecessary in the example,
% but is helpful to illustrate the type-checking against the
% return value of the call to append.

A warning is issued
for the line that calls \code{append(".txt")} because the value of
\code{filename} is guarded by \code{lock}, whereas the formal receiver parameter
\code{StringBuffer this} is \code{@GuardedBy(\{\})}.  The \code{append}
method needs to be annotated to prevent this warning.

The documentation for \sunjavadoc{java/lang/StringBuffer.html\#append-java.lang.String-}{StringBuffer.append(String str)}
indicates that the method returns the updated receiver.  In this case,
it is not correct to annotate the receiver formal parameter with
\code{@NonLeakedUnknownGuard}, because the receiver is leaked to the
return value.  Instead, the \code{@LeakedToResultUnknownGuard} annotation is needed:

\begin{verbatim}
@LockingFree
public StringBuffer append(@LeakedToResultUnknownGuard StringBuffer this, String str)
\end{verbatim}

With this \code{@LeakedToResultUnknownGuard} annotation in place, the
Lock Checker now issues no warnings for the \code{append(".txt")}
method call in the user code.  This is because (1) \code{lock}
is known to be held prior to the method call, (2) the
value in \code{filename} is known not to be leaked except to the return
value and possibly local variables, and (3) the method
is annotated with \code{@LockingFree}, therefore it is not possible
for \code{lock} to be unlocked in the middle of the method execution
before \code{filename} is dereferenced by the method.  The
user sample code now type-checks without warnings.

Based on the method documentation, it is possible to annotate it more generally as:

\begin{verbatim}
@LockingFree
public StringBuffer append(@LeakedToResultUnknownGuard StringBuffer this, @NonLeakedUnknownGuard String str)
\end{verbatim}

Note that the return type may be also annotated with \code{@LeakedToResultUnknownGuard},
but it is not necessary, as that is implied by the presence of the \code{@LeakedToResultUnknownGuard}
annotation in one of the parameters.  Also, if more than one formal parameter
is annotated with \code{@LeakedToResultUnknownGuard}, the \code{@GuardedBy}
annotations at the method call site must match exactly for all the corresponding
actual parameters, and these must of course also exactly match the \code{@GuardedBy}
annotation on the return type.

Note that \code{@NonLeakedUnknownGuard} and \code{@LeakedToResultUnknownGuard}
have distinct meanings from \code{@GuardedBy(\{\})}.  \code{@GuardedBy(\{\})}
means that it is \emph{known} that a value can always be safely dereferenced
regardless of the lock expressions that are held on the current thread.

\paragraph{Summary\label{lock-checker-library-methods-summary}}

The \code{@LockingFree}, \code{@NonLeakedUnknownGuard}, and
\code{@LeakedToResultUnknownGuard} annotations allow the programmer
to annotate certain library methods that do not make any use of
locks or synchronization such that the Lock Checker issues no
warnings when calls are made to these methods when all the
appropriate lock expressions are held.  This helps reduce the
overall annotation burden for the programmer.


\section{Examples\label{lock-examples}}

The Lock Checker guarantees that a value that was computed from an expression of \code{@GuardedBy} type is
never dereferenced unless all the expressions indicated in the
\code{@GuardedBy} annotation are held by the thread performing the
dereference, at the time of the dereference.  This is true regardless of how
the value was dereferenced (via a variable, via the result of a method call, etc.).

In the following example, a value is created via the expression
\code{new Object()}.  The example shows where the Lock Checker issues errors
for dereferences of the value that are not permitted.  The Lock Checker
detects these illegal dereferences even though the value is stored in
different variables and is returned by a method call.

Note that the expression \code{new Object()} is assumed to have type \code{@GuardedBy("lock")}
because it is immediately assigned to a newly declared
variable having type annotation \code{@GuardedBy("lock")}.  You could
explicitly write \code{new @GuardedBy("lock") Object()} but it is not
required.

\begin{Verbatim}
class MyClass {
  Object lock; // Initialized in the constructor

  @GuardedBy("lock") Object x = new Object();
  @GuardedBy("lock") Object y = x; // OK, because dereferences of y will require "lock" to be held.
  @GuardedBy({}) Object z = x; // ILLEGAL because dereferences of z do not require "lock" to be held.
  @GuardedBy("lock") Object myMethod(){
     return x; // OK because the return type is annotated with @GuardedBy("lock")
  }

  [...]

  void exampleMethod(){
     x.toString(); // ILLEGAL because the lock is not known to be held
     y.toString(); // ILLEGAL because the lock is not known to be held
     myMethod().toString(); // ILLEGAL because the lock is not known to be held
     synchronized(lock) {
       x.toString();  // OK: the lock is known to be held
       y.toString();  // OK: the lock is known to be held
       myMethod().toString(); // OK: the lock is known to be held
     }
  }
}
\end{Verbatim}


\subsection{Examples of @GuardedBy and @Holding\label{lock-examples-guardedby-and-holding}}

\textbf{@GuardedBy(Eset)}

The following example demonstrates the reason the Lock Checker enforces the
following rule:  if \code{Eset1} $\neq$ \code{Eset2}, then
\code{@GuardedBy(Eset1)} and \code{@GuardedBy(Eset2)} are siblings in the type
hierarchy.  Suppose the Lock Checker issued no errors for the following code:

\begin{Verbatim}
class MyClass {
    Object a = new Object();
    Object b = new Object();
    @GuardedBy("a") Object x = new Object();
    @GuardedBy({"a", "b"}) Object y = new Object();
    void myMethod() {
        y = x;
    }
}
\end{Verbatim}

At first glance, it appears that the assignment should be allowed, since
whenever \code{y}'s value is dereferenced, locks \code{a} and \code{b} are
held, so the guarantee about \code{x} is satisfied.  However, suppose the
following code can be executed after the assignment \code{y = x}:

\begin{Verbatim}
synchronized(a) {
  x.toString();
}
\end{Verbatim}

The problem is that variables \code{x} and \code{y} now contain references to
the same object, and by dereferencing \code{x}, the code is also dereferencing
\code{y}.  However the programmer requested that any value stored in variable
\code{y} only be dereferenced when both \code{a} and \code{b} are held.

To prevent this kind of problem, the Lock Checker forbids the assignment
\code{y = x} by enforcing the aforementioned rule.
\\

The following example further demonstrates the enforcement of this rule:

\begin{Verbatim}
@GuardedBy({}) Object o1;
@GuardedBy("lock") Object o2;
@GuardedBy("lock") Object o3;
[...]
o3 = o2; // OK, since o2 and o3 are guarded by exactly the same lock set.

o1 = o2; // Assignment type incompatible errors are issued for both assignments, since
o2 = o1; // {"lock"} and {} are not identical sets.

@SuppressWarnings("lock:cast.unsafe")
o1 = (@GuardedBy({}) Object) o2; // A cast can be used if the user knows it is safe to do so.
                                 // However the @SuppressWarnings must be added.
\end{Verbatim}





The most common use of \code{@GuardedBy} is to annotate a field declaration
type.  However, other uses of \code{@GuardedBy} are possible.

\paragraph{Return types\label{lock-examples-guardedby-return}}

A return type may be annotated with \code{@GuardedBy}:

\begin{Verbatim}
  @GuardedBy("MyClass.myLock") Object myMethod() { ... }

  // reassignments without holding the lock are OK.
  @GuardedBy("MyClass.myLock") Object x = myMethod();
  @GuardedBy("MyClass.myLock") Object y = x;
  x.toString(); // ILLEGAL because the lock is not held
  synchronized(MyClass.myLock) {
    y.toString();  // OK: the lock is held
  }
\end{Verbatim}

\paragraph{Formal parameters\label{lock-examples-guardedby-formal-parameters}}

A parameter type may be annotated with \code{@GuardedBy}, indicating that
the method body must acquire the lock before dereferencing the parameter.

\begin{Verbatim}
 void myMethod(@GuardedBy("MyClass.myLock") Object a) {
    a.toString(); // ILLEGAL: the lock is not held
    synchronized(MyClass.myLock) {
      a.toString();  // OK: the lock is held
    }
  }
\end{Verbatim}

\textbf{Example of @GuardedBy and @Holding}

The following example shows the interaction between these two annotations:

\begin{Verbatim}
  void helper1(@GuardedBy("MyClass.myLock") Object a) {
    a.toString(); // ILLEGAL: the lock is not held
    synchronized(MyClass.myLock) {
      a.toString();  // OK: the lock is held
    }
  }
  @Holding("MyClass.myLock")
  void helper2(@GuardedBy("MyClass.myLock") Object b) {
    b.toString(); // OK: the lock is held
  }
  void helper3(@GuardedBy("MyClass.myLock") Object d) {
    d.toString(); // ILLEGAL: the lock is not held
  }
  void myMethod2(@GuardedBy("MyClass.myLock") Object e) {
    helper1(e);  // OK to pass to another routine without holding the lock
    e.toString(); // ILLEGAL: the lock is not held
    synchronized (MyClass.myLock) {
      helper2(e);
      helper3(e); // OK, but helper3's body still has an error.
    }
  }
\end{Verbatim}

\subsection{Examples of @EnsuresLockHeld and @EnsuresLockHeldIf\label{ensureslockheld-examples}}

\code{@EnsuresLockHeld} and \code{@EnsuresLockHeldIf} are primarily intended
for annotating JDK locking methods, as in:

\begin{Verbatim}
package java.util.concurrent.locks;

class ReentrantLock {

    @EnsuresLockHeld("this")
    public void lock();

    @EnsuresLockHeldIf (expression="this", result=true)
    public boolean tryLock();

[...]
}
\end{Verbatim}

They can also be used to annotate user methods, particularly for
higher-level lock constructs such as a Monitor, as in this simplified example:

\begin{Verbatim}
public class Monitor {

    private ReentrantLock lock; // Initialized in the constructor

[...]

    @EnsuresLockHeld("lock")
    public void enter() {
       lock.lock();
    }

[...]
}
\end{Verbatim}

\subsection{Example of @LockingFree\label{lock-lockingfree-example}}

\code{@LockingFree} is useful when a method does not make any use of synchronization
or locks but causes other side effects (hence \code{@SideEffectFree} is not appropriate).
\code{@SideEffectFree} implies \code{@LockingFree}, therefore if both are applicable,
you should only write \code{@SideEffectFree}.


\begin{verbatim}
private Object myField;
private ReentrantLock lock; // Initialized in the constructor
private @GuardedBy("lock") Object x; // Initialized in the constructor

[...]

// This method does not use locks or synchronization but cannot
// be annotated as @SideEffectFree since it alters myField.
@LockingFree
void myMethod() {
    myField = new Object();
}

@SideEffectFree
int mySideEffectFreeMethod() {
    return 0;
}

void myUnlockingMethod() {
    lock.unlock();
}

void myUnannotatedEmptyMethod() {
}

void myOtherMethod() {
    if (lock.tryLock()) {
        x.toString(); // OK: the lock is held
        myMethod();
        x.toString(); // OK: the lock is still known to be held since myMethod is locking-free
        mySideEffectFreeMethod();
        x.toString(); // OK: the lock is still known to be held since mySideEffectFreeMethod
                      // is side-effect-free
        myUnlockingMethod();
        x.toString(); // ILLEGAL: myUnlockingMethod is not locking-free
    }
    if (lock.tryLock()) {
        x.toString(); // OK: the lock is held
        myUnannotatedEmptyMethod();
        x.toString(); // ILLEGAL: even though myUnannotatedEmptyMethod is empty, since it is
                      // not annotated with @LockingFree, the Lock Checker no longer knows
                      // the state of the lock.
    }
}
\end{verbatim}




\section{More locking details\label{lock-details}}

\subsection{Two types of locking:  monitor locks and explicit locks\label{lock-two-types}}

Java provides two types of locking:  monitor locks and explicit locks.

\begin{itemize}
\item
  A \<synchronized(\emph{E})> block acquires the lock on the value of
  \emph{E}; similarly, a method declared using the \<synchronized> method
  modifier acquires the lock on the method receiver when called.
  (More precisely,
  the current thread locks the monitor associated with the value of
  \emph{E}; see \href{https://docs.oracle.com/javase/specs/jls/se8/html/jls-17.html#jls-17.1}{JLS \S17.1}.)
  The lock is automatically released when execution exits the block or the
  method body, respectively.
  We use the term ``monitor lock'' for a lock acquired using a
  \<synchronized>  block or \<synchronized> method modifier.
\item A method call, such as
  \sunjavadoc{java/util/concurrent/locks/Lock.html\#lock()}{Lock.lock()},
  acquires a lock that implements the
  \sunjavadoc{java/util/concurrent/locks/Lock.html}{Lock}
  interface.
  The lock is released by another method call, such as
  \sunjavadoc{java/util/concurrent/locks/Lock.html\#unlock()}{Lock.unlock()}.
  We use the term ``explicit lock'' for a lock expression acquired in this
  way.
\end{itemize}

You should not mix the two varieties of locking, and the Lock Checker
enforces this.  To prevent an object from being used both as a monitor and
an explicit lock, the Lock Checker issues a warning if a
\<synchronized(\emph{E})> block's expression \<\emph{E}> has a type that
implements \sunjavadoc{java/util/concurrent/locks/Lock.html}{Lock}.
% The Lock Checker does not keep track of which locks are monitors
% and which are explicit, so this check is necessary for the Lock Checker to
% function correctly, and it also alerts the programmer of a code smell.


\subsection{Held locks and held expressions; aliasing\label{lock-aliasing}}

Whereas Java locking is defined in terms of values, Java programs are
written in terms of expressions.
We say that a lock expression is held if the value to which the expression
currently evaluates is held.

The Lock Checker conservatively estimates the expressions that are held at
each point in a program.
The Lock Checker does not track aliasing
(different expressions that evaluate to the same value); it only considers
the exact expression used to acquire a lock to be held.  After any statement
that might side-effect a held expression or a lock expression, the Lock
Checker conservatively considers the expression to be no longer held.

Section~\ref{java-expressions-as-arguments} explains which Java
expressions the Lock Checker is able to analyze as lock expressions.


\section{Other lock annotations\label{lock-other-annotations}}

The Checker Framework's lock annotations are similar to annotations used
elsewhere.

If your code is already annotated with a different lock
annotation, you can reuse that effort.  The Checker Framework comes with
cleanroom re-implementations of annotations from other tools.  It treats
them exactly as if you had written the corresponding annotation from the
Lock Checker, as described in Figure~\ref{fig-lock-refactoring}.


% These lists should be kept in sync with LockAnnotatedTypeFactory.java .
\begin{figure}
\begin{center}
% The ~ around the text makes things look better in Hevea (and not terrible
% in LaTeX).

\begin{tabular}{ll}
\begin{tabular}{|l|}
\hline
 ~net.jcip.annotations.GuardedBy~ \\ \hline
 ~javax.annotation.concurrent.GuardedBy~ \\ \hline
\end{tabular}
&
$\Rightarrow$
~org.checkerframework.checker.lock.qual.GuardedBy~
\end{tabular}
\end{center}
%BEGIN LATEX
\vspace{-1.5\baselineskip}
%END LATEX
\caption{Correspondence between other lock annotations and the
  Checker Framework's annotations.}
\label{fig-lock-refactoring}
\end{figure}

Alternately, the Checker Framework can process those other annotations (as
well as its own, if they also appear in your program).  The Checker
Framework has its own definition of the annotations on the left side of
Figure~\ref{fig-lock-refactoring}, so that they can be used as type
annotations.  The Checker Framework interprets them according to the right
side of Figure~\ref{fig-lock-refactoring}.


\subsection{Relationship to annotations in \emph{Java Concurrency in Practice}\label{lock-jcip-annotations}}

The book \href{http://jcip.net/}{\emph{Java Concurrency in Practice}}~\cite{Goetz2006} defines a
\href{http://jcip.net.s3-website-us-east-1.amazonaws.com/annotations/doc/net/jcip/annotations/GuardedBy.html}{\code{@GuardedBy}} annotation that is the inspiration for ours.  The book's
\code{@GuardedBy} serves two related but distinct purposes:

\begin{itemize}
\item
  When applied to a field, it means that the given lock must be held when
  accessing the field.  The lock acquisition and the field access may be
  arbitrarily far in the future.
\item
  When applied to a method, it means that the given lock must be held by
  the caller at the time that the method is called --- in other words, at
  the time that execution passes the \code{@GuardedBy} annotation.
\end{itemize}

The Lock Checker renames the method annotation to
\refqualclass{checker/lock/qual}{Holding}, and it generalizes the
\refqualclass{checker/lock/qual}{GuardedBy} annotation into a type annotation
that can apply not just to a field but to an arbitrary type (including the
type of a parameter, return value, local variable, generic type parameter,
etc.).  This makes the annotations more expressive and also more amenable
to automated checking.  It also accommodates the distinct
meanings of the two annotations, and resolves ambiguity when \<@GuardedBy>
is written in a location that might apply to either the method or the
return type.

(The JCIP book gives some rationales for reusing the annotation name for
two purposes.  One rationale is
that there are fewer annotations to learn.  Another rationale is
that both variables and methods are ``members'' that can be ``accessed'';
variables can be accessed by reading or writing them (putfield, getfield),
and methods can be accessed by calling them (invokevirtual,
invokeinterface):  in both cases, \code{@GuardedBy} creates preconditions
for accessing so-annotated members.  This informal intuition is
inappropriate for a tool that requires precise semantics.)

% It would not work to retain the name \code{@GuardedBy} but put it on the
% receiver; an annotation on the receiver indicates what lock must be held
% when it is accessed in the future, not what must have already been held
% when the method was called.


\section{Possible extensions\label{lock-extensions}}

The Lock Checker validates some uses of locks, but not all.  It would be
possible to enrich it with additional annotations.  This would increase the
programmer annotation burden, but would provide additional guarantees.

Lock ordering:  Specify that one lock must be acquired before or after
another, or specify a global ordering for all locks.  This would prevent
deadlock.

Not-holding:  Specify that a method must not be called if any of the listed
locks are held.

These features are supported by
\href{http://clang.llvm.org/docs/ThreadSafetyAnalysis.html}{Clang's
  thread-safety analysis}.


\section{Annotations used internally by the Lock Checker\label{lock-internals}}

The \code{@LockHeld} and \code{@LockPossiblyHeld} type qualifiers are used internally by the Lock Checker
and should never be written by the programmer.
If you
are seeing a warning mentioning \code{@LockHeld} or \code{@LockPossiblyHeld},
please contact the Checker Framework developers as it is likely to
indicate a bug in the Checker Framework.


% LocalWords:  quals GuardedBy JCIP putfield getfield invokevirtual 5cm
% LocalWords:  invokeinterface threadsafety Clang's GuardedByInaccessible cleanroom
%%  LocalWords:  api 5cm
